Dans le cadre du cours d'Électronique pour ASI, nous avons dû réaliser un projet de notre choix. Il nous a été proposé de reprendre un projet de l'année dernière ou de concevoir notre propre projet. Nous avons choisi de reprendre le projet de deux de nos camarades de l'année dernière. La durée approximative de ce projet était de 6 mois.\\

Notre groupe était composé de cinq membres : Antoine Augusti, Étienne Batise, Jean-Claude Bernard et de Thibaud Dauce. Le projet, que nous avions choisi, était le réveil intelligent. Celui-ci fonctionnait à l'aide du couplage de la carte Arduino et de la RaspberryPi. Grâce à cette interaction, le réveil sonnait à une certaine heure et  des informations s'affichaient à l'écran. Notre objectif a été de reprendre les outils qu'ils nous avaient laissés pour améliorer le réveil.\\

Dans ce rapport, nous abordons les nouveautés que nous avons apportées ainsi que les démarches effectuées pour mener le projet à son terme. Ainsi, dans une première partie nous présenterons le cahier des charges, suivi par les études et recherches menées ainsi que les documents de réalisation associés, pour enfin terminer par le bon de commande et le bilan de notre projet.
