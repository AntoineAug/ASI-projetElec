\section{Introduction en français}
Dans le cadre du cours d'électronique pour ASI, nous avons dû réaliser un projet de notre choix. Il nous a été proposé de reprendre un projet de l'année dernière ou de concevoir notre propre projet. Nous avons choisi de reprendre le projet de deux de nos camarades de l'année dernière. La durée approximative de ce projet était de 6 mois.\\

Notre groupe était composé de quatre membres : Antoine Augusti, Étienne Batise, Jean-Claude Bernard et de Thibaud Dauce. Le projet, que nous avions choisi, était le réveil intelligent. Celui-ci fonctionnait à l'aide du couplage de la carte Arduino et de la Raspberry Pi. Grâce à cette interaction, le réveil sonnait à une certaine heure et  des informations s'affichaient à l'écran. Notre objectif a été de reprendre les outils qu'ils nous avaient laissés pour améliorer le réveil.\\

Dans ce rapport, nous abordons les nouveautés que nous avons apportées ainsi que les démarches effectuées pour mener le projet à son terme. Ainsi, dans une première partie nous présenterons le cahier des charges, suivi par les études et recherches menées ainsi que les documents de réalisation associés, pour enfin terminer par le bon de commande et le bilan de notre projet.

\section{Summary of our project}
We realized this project as part of our electronics course at the INSA de Rouen. We chose to continue the project made by two friends, last year. Our project is called \textit{a smart alarm clock} and our goal was to build an alarm clock that wakes you up peacefully in the morning by playing some music and turning on the light gradually. Furthermore, our alarm clock has got a nice web interface where you can glance at the news and see the coming subways when you wake up in the morning.\\

In order to achieve this goal, we had to use many awesome technologies: a Raspberry Pi, an Arduino and a lot of programming languages (Bash, PHP, HTML, CSS, JavaScript, Arduino). The main difficulty was to know how to use and combine those technologies together.\\

In this report, written in French, you will be able to see how we manage to achieve our project and built our \textit{smart	alarm clock}.