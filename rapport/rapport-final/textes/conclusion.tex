La réalisation de ce projet d'électronique nous a permis tout d'abord d'acquérir des compétences dans de multiples domaines : l'électronique, la programmation et le réseau. Nous avons dû faire appel à de nombreuses connaissances et nous avons effectué de nombreuses recherches pour pouvoir réaliser l'évolution du réveil intelligent, particulièrement en électronique pour la Raspberry et la Arduino.\\

Bien que nous étions à travailler en groupe avec les précédents projets que nous avons pu réalisé lors de notre scolarité à l'INSA, ce projet a été différent car pour la première fois nous devions rendre un objet au terme du projet.\\

La partie qui nous a demandé le plus de travail a été l'interaction entre les différentes parties de notre projet : l'interface web de configuration, le système d'exploitation de la Raspberry ainsi que la communication entre la Raspberry et la Arduino.\\

Bien que notre projet puisse être amélioré et que celui ne soit absolument pas présentable en tant que réveil intelligent complet, nous sommes satisfaits du travail que nous avons fourni car nous avons rencontré divers obstacles avant d'arriver au résultat actuel.