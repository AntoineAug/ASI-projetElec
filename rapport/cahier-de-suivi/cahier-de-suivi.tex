\documentclass[a4paper, 12pt, french]{article}

\usepackage[utf8]{inputenc} 	%package pour le français sous ubuntu : à vous d'adapter
\usepackage[french]{babel}	%pour le français
\usepackage[T1]{fontenc}	%pour les polices
\usepackage{fullpage}

\title{Cahier de suivi}
\author{Antoine Augusti\\ Etienne Batise\\ Jean-Claude Bernard\\ Thibaud Dauce}

\begin{document}
\maketitle

	\section{Mercredi 09 Octobre 2013} % (fold)
	\label{sec:mercredi_09_octobre_2013}
	\begin{itemize}
		\item Développement du programme Arduino. À faire : les tests.
		\item Le shield actuel a été refait et validé. Il reste à monter la nouvelle ampoule dessus.
		\item Recherches sur l'ampoule ainsi que sur la méthode d'alimentation autonome(Batterie Li-Po x2, Régulateur de tension 7805, cartes montées en parallèles et ampoules 4,8V)
		\item
	\end{itemize}
	% section mercredi_09_octobre_2013 (end)

	\section{Mercredi 25 Septembre 2013} % (fold)
	\label{sec:mercredi_25_septembre_2013}
	\begin{itemize}
			\item Prise en main de l'interface du RaspberryPi. Beaucoup de problème avec le réseau INSA (proxy vs serveur local)
			\item Prise en main de l'arduino et lecture du code.
			\item Définition des objectifs à atteindre du point du vue électronique..
	\end{itemize}	
	% section mercredi_25_septembre_2013 (end)
\end{document}