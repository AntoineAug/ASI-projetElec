\documentclass[a4paper, 12pt, french]{article}

\usepackage[utf8]{inputenc} 	%package pour le français sous ubuntu : à vous d'adapter
\usepackage[french]{babel}	%pour le français
\usepackage[T1]{fontenc}	%pour les polices
\usepackage{fullpage}
\usepackage{array}

\title{Calendrier previsionnel}
\author{Antoine Augusti\\ Etienne Batise\\ Jean-Claude Bernard\\ Thibaud Dauce}

\begin{document}
	
	\maketitle

	\centering{
	\begin{tabular}{| l | m{3.5cm} | m{3.5cm} | m{3.5cm} |}
	
	\hline
	\textbf{Dates} & \textbf{Électronique} & \textbf{Arduino} & \textbf{RaspberryPi} \\ \hline
	25/09/2013 & Determiner le travail à fournir & Prendre en main le logiciel Arduino & Découvrir l'interface RaspberryPi \\ \hline
	
	09/10/2013 & Remplacer le Shield Arduino par une carte PCB & Commencer le développement de la gestion de l'ampoule & \\ \hline
	
	23/10/2013 & 
				\begin{itemize}
				\item Ajouter une ampoule au montage
				\item Trouver une solution pour une alimentation autonome
				\end{itemize}	& Effectuer les tests de la gestion de l'ampoule & \\ \hline
	
	06/11/2013 & Implémentation de l'alimentation autonome & Développer un indicateur de la charge restante de la batterie & Commencer le développement du gestionnaire de métros \\ \hline
	
	20/11/2013 & Effectuer et vérifier tous les branchements définitevement & & Commencer le développement du gestionnaire de sons\\ \hline
	
	04/12/2013 & & & Commencer la gestion d'un affichage sur écran LCD \\ \hline
	
	18/12/2013 & \multicolumn{3}{c|}{Bricolage} \\ \hline	
	
	\end{tabular}
	}
\end{document}